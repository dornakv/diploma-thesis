%% First of all, include a document class and options
\documentclass[english,master]{diploma}

%% Includes
\usepackage[autostyle=true]{csquotes} % enhanced support for quotation marks, support for biblatex package
\usepackage[backend=biber, style=iso-numeric, alldates=iso]{biblatex} % bibliography
\DeclareNameAlias{default}{family-given} % Name format 'last-first' deprecated. Use 'family-given'
\usepackage{csvsimple}
\usepackage{amsmath}
\usepackage{bm}
\usepackage{amsfonts}
\usepackage{mathtools}
\mathtoolsset{showonlyrefs=true}
\usepackage{xfrac}
\usepackage{caption}
\usepackage{subcaption}
\usepackage{dirtytalk}
\usepackage{microtype}
\usepackage{multirow, makecell}
\emergencystretch=1em
\newcommand*\Laplace{\mathop{}\!\mathbin\bigtriangleup}
\newcommand{\figref}[1]{\figurename~\ref{#1}}
\newcommand{\tabref}[1]{\tablename~\ref{#1}}
\newcommand{\norm}[1]{\left\lVert#1\right\rVert}
\DeclareMathOperator*{\argmin}{arg\,min}
\DeclareMathOperator*{\argmax}{arg\,max}

\usepackage{xcolor}
\usepackage{colortbl}
\definecolor{LightCyan}{rgb}{0.82,1,1}
\definecolor{Cyan}{rgb}{0.60,1,1}

%% Next, enter data for leading pages
\ThesisAuthor{Bc. Vojtěch Dorňák}

\ThesisSupervisor{Ing. Lukáš Pospíšil, Ph.D.}

\CzechThesisTitle{Algoritmy pro klasifikaci visuálních signálů s využitím technik extrakce významných rysů}

\EnglishThesisTitle{Algorithms used for classifying visual signals using methods for extracting significant features}

\SubmissionYear{2021}

\Acknowledgement{I would like to thank all those who helped me with the work,
because without them this work would not have happened.}

\CzechAbstract{Klasifikace vizuálních dat je důležitý úkol v moderním oboru počítačového vidění. Přestože je populární řešit tento úkol pomocí neuronových sítí, my prozkoumáme jiný přístup. Projektujeme vizuální data do prostoru rysů, a pokusíme se tato transformovaná data klasifikovat konvenčnějšími klasifikačními metodami. Pro transformaci použijeme SIFT, SURF, ORB, a PCA extraktory. Porovnáme dva klasifikátory, SVM a Bayesovský model. Experimentujeme s těmito přístupy na třech různě komplexních datasetech.}

\CzechKeywords{klasifikace vizuálních dat, extrakce rysů, Support Vector Machine, Bag of Words, k-means, Scale-Invariant Feature Transform, Speed-Up Robust Features, Oriented Fast and Rotated Brief, Analýza hlavních komponent}

\EnglishAbstract{Classification of visual data is an important task in the modern computer vision field. While it is popular to address this task using neural networks, we explore a different approach. We project the visual data into a space of features, and attempt to classify the projected data using more conventional classification methods. For the feature transformation, we use SIFT, SURF, ORB, and PCA extractors. We compare two classification methods, i.e. the SVM and the Bayesian Model. We experiment with these approaches on three datasets of different complexity.}

\EnglishKeywords{visual data classification, feature extraction, Support Vector Machine, Bag of Words, k-means, Scale-Invariant Feature Transform, Speed-Up Robust Features, Oriented Fast and Rotated Brief, Principal Component Analysis}

\AddAcronym{SIFT}{Scale Invariant Feature Transform}
\AddAcronym{SURF}{Speed Up Robust Features}
\AddAcronym{ORB}{Oriented Fast and Rotated Brief}
\AddAcronym{SVM}{Support Vector Machine}
\AddAcronym{PCA}{Principal Component Analysis}
\AddAcronym{BoVW}{Bag of Visual Words}

% Bibliography resources for BibLaTeX
\addbibresource{main.bib}

%% Beginning of the document
\begin{document}

%% Leading pages printing
\MakeTitlePages

\listoffigures
\clearpage

\listoftables
\clearpage

\chapter{Introduction}
The goal of this thesis...


\chapter{Image Data Transformation}

We can attempt to classify the raw image data where the image is represented as a vector of light intensities at each pixel. However, it might be beneficial to transform the image data into a feature space using a feature extraction technique. The feature extraction techniques can be split into two categories: local feature extractors and global feature extractors\cite{lee2005}.

Local feature extractors localize such points in an image, which could be found regardless of the image subject position. These points are called key-points. The area around such key-points is then described using a vector called a descriptor. The descriptors are created in a way, which allows for matching similar features.

Global feature extractors transform the whole image into a lower-dimensional vector, attempting to retain the most important information.

For feature extraction, we compare three local feature extractors: SIFT, SURF and ORB, and a global feature extractor: PCA.

In this section, we describe the four feature extraction algorithms.

\section{Scale Invariant Feature Transform}
Scale Invariant Feature Transform (SIFT) was proposed by David Lowe, and published in the original paper \cite{Lowe1999} in 1999. Compared to ordinary techniques such as the Harris Corner Detector \cite{Harris1988} or Canny edge detector \cite{Canny1986}, the SIFT identifies the general (not only edges and corners) key-points. The descriptors, which describe the local image region around the key-points are scale-invariant. Moreover, they are invariant to rotation, illumination, and change in affine transformations. In our text, we introduce the methodology of training the classifier that recognizes the objects in real images (photographs); therefore the properties of SIFT feature vector are essential.

\subsection{Key-point Detection}

Let us consider an input image \( I_{img}(x,y) \). A convolution of the image with a Gaussian kernel
\begin{equation}
    G(x,y,\sigma) = \frac{1}{2\pi\sigma^2}e^{-\frac{x^2+y^2}{2\sigma^2}},
    \label{eq:Gaussian_kernel}
\end{equation}
at a scale \( \sigma > 0 \) is exploited to get a scale-space representation of the image so that:
\begin{equation}
    L(x, y,\sigma) =  G(x,y,\sigma)*I_{img}(x,y).
\end{equation}

To detect scale-invariant key-points, a scale normalized Laplacian of Gaussian (LoG)
\begin{equation}
    \Laplace_{norm} L(x, y, \sigma) = \sigma\left(\frac{\partial^2 L(x,y,\sigma)}{\partial x^2} + \frac{\partial^2 L(x,y,\sigma)}{\partial y^2}\right)
\end{equation}
is required \cite{Koenderink1984}. It has been shown \cite{Mikolajczyk2002}, that the local extrema of LoG provide the most stable image features, compared to many other popular image functions.

Determination of the LoG would be time-consuming, therefore, it is approximated by the Difference of Gaussian (DoG) \cite{Lowe2004}. The DoG is computed from two adjacent scales separated by a constant multiplicative factor $k \in \mathbb{R}$:
\begin{equation}
    D(x,y,\sigma) := L(x,y,k\sigma) - L(x,y,\sigma).
\end{equation}

To ensure the scale invariance, the extrema are located not only in the domain of one DoG, however, it is found across different scales as well. The different DoGs at the scales are constructed by progressively convolving the original image with the Gaussian kernel. The consecutive Gaussian kernels' scale differs by the multiplicative factor $k$.

At each doubling of the scale $k$, the resolution of an image can be reduced by a factor of $2$ to provide better efficiency. Each group of blurred images of the same resolution is called an octave. In each octave, we generate the DoG by subtracting the $L(x, y, \sigma)$ of neighboring scales. This is called the DoG pyramid and can be seen in \figref{fig:DoG_pyramid}.

\begin{figure}
    \centering
    \includegraphics[width=0.8\textwidth]{Figures/sift/pyramid.jpg}
    \caption[DoG pyramid]{DoG pyramid \cite{Lowe2004}.}
    \label{fig:DoG_pyramid}
\end{figure}

Each pixel in the DoG pyramid is compared to the $3\times3$ pixels in DoGs below and above, and $8$ surrounding pixels. This is shown in \figref{fig:DoG_extrema}. The pixel is selected as a key-point candidate if it has a lower or a higher value than all of these $26$ pixels.

\begin{figure}
    \centering
    \includegraphics[width=0.4\textwidth]{Figures/sift/extrema.jpg}
    \caption[Optima of the DoG are selected by the means of comparing pixel to the 26 pixels surrounding it in the scale-space]{Optima of the DoG are selected by the means of comparing pixel to the 26 pixels surrounding it in the scale-space \cite{Lowe2004}.}
    \label{fig:DoG_extrema}
\end{figure}

To detect sub-pixel locations of extrema, the DoG is interpolated using quadratic Taylor expansion at each key-point candidate:
\begin{equation}
    D(\boldsymbol{x}) = D + \frac{\partial D^T}{\partial \boldsymbol{x}}\boldsymbol{x}+\frac{1}{2}\boldsymbol{x}^T\frac{\partial^2 D}{\partial \boldsymbol{x}^2}\boldsymbol{x},
    \label{eq:taylor_expansion}
\end{equation}
where \( D(x,y,\sigma) \) and the partial derivatives are evaluated at the key-point and \( \boldsymbol{x}=(x,y,\sigma) \) is the offset from the key-point. Then, extremum \( \hat{\boldsymbol{x}} \) is located by setting the gradient of \( D \) to be a zero vector:
\begin{equation}
    \hat{\boldsymbol{x}} = - \frac{\partial^2 D^{-1}}{\partial \boldsymbol{x}^2}\frac{\partial D}{\partial \boldsymbol{x}}.
    \label{eq:taylor_extremum}
\end{equation}
If the offset \( \hat{\boldsymbol{x}} \) is larger than \( 0.5 \) in any dimension, it indicates that the extremum is closer to another point. In this case, the candidate point is changed to the new point, and interpolation is performed again.

The function value at the sub-pixel extrema ,$D(\hat{\boldsymbol{x}})$, is used for rejecting extrema with low contrast. In our experiments, we use the OpenCV SIFT implementation, where the default value is $\frac{0.04}{3}$ \cite{openCV}.

The DoG has a strong response along edges, even if the location along the edge is poorly determined. These candidate key-points have principal curvature perpendicular to the edge much larger than the principal curvature along it. The principal curvatures can be computed from a $2\times2$ Hessian matrix at the location of the key-point candidate
\begin{equation}
    \boldsymbol{H} =
    \begin{bmatrix}
        D_{xx} & D_{xy}\\
        D_{xy} & D_{yy}
    \end{bmatrix},
    \label{eq:hessian}
\end{equation}
where the partial derivatives are estimated by taking the differences of neighboring sample points. The eigenvalues of $\boldsymbol{H}$ are proportional to the principal curvatures.

The computation of eigenvalues can be avoided, as only the ratio of the eigenvalues is required. Let us denote the larger eigenvalue as $\alpha$ and the smaller one as $\beta$. The trace of $\boldsymbol{H}$ is defined as
\begin{equation}
    Tr(\boldsymbol{H}) = D_{xx}+D_{yy} = \alpha+\beta,
\end{equation}
and the determinant as
\begin{equation}
    Det(\boldsymbol{H}) = D_{xx}D_{yy}-D_{xy}^2 = \alpha\beta.
\end{equation}

Let $r$ be the ratio of the eigenvalues, such that
\begin{equation}
    r=\frac{\alpha}{\beta}.
\end{equation}
Then,
\begin{equation}
    \frac{Tr(H)^2}{Det(\boldsymbol{H})} = \frac{(\alpha+\beta)^2}{\alpha\beta}=\frac{(r\beta+\beta)^2}{r\beta^2} = \frac{(r+1)^2}{r}.
\end{equation}

Therefore, to inspect the ratio of eigenvalues, we can check
\begin{equation}
    \frac{Tr(H)^2}{Det(\boldsymbol{H})} = \frac{(\alpha+\beta)^2}{\alpha\beta}=\frac{(r\beta+\beta)^2}{r\beta^2} = \frac{(r+1)^2}{r}.
\end{equation}
The candidate key-points with this ratio below a threshold $r$ is discarded. In our experiments, we use the default value of OpenCV SIFT implementation, which is equal to $10$.

\subsection{Key-point Description}
We want to create a descriptor for each of our key-points. These descriptors must be almost identical across different scales, rotations, illuminations, and other transformations.

To ensure descriptor invariance to a rotation, we first determine the key-point orientation. We calculate the gradient magnitude
\begin{equation}
    m(x,y) = \sqrt{(L(x+1,y)-L(x-1,y))^2+(L(x,y+1)-L(x,y-1))^2},
\end{equation}
and orientation
\begin{equation}
    \theta(x,y) = tan^{-1}\left(\frac{L(x,y+1)-L(x,y-1)}{L(x+1,y)-L(x-1,y)}\right),
\end{equation}
of each point within a region around the key-point. From these values, an orientation histogram with $36$ bins is created. The largest peak is selected as a key-point orientation. If there are other peaks more than $80\%$ of the largest peak, new key-points are created at the same location with the other peaks as their orientation.

An example of key-points and their orientation, found by OpenCV SIFT implementation, can be seen in \figref{fig:sift_example}.
\begin{figure}[ht!]
    \centering
    \includegraphics[width=0.55\textwidth]{Figures/sift/sift_example.jpg}
    \caption[SIFT key-points and their orientation from an image of a cat]{SIFT key-points and their orientation from an image of a cat.}
    \label{fig:sift_example}
\end{figure}

The descriptor is constructed from a window of the size $16\times16$ pixels around the key-point. This window is rotated by the orientation of the key-point, which ensures the rotation invariance of the keypoint. In this window, the gradient magnitude and the orientation are computed for each point. The $16\times16$ window is divided into $16$ ($4\times4$) sub-windows. For each sub-window, an $8$ bin gradient orientation histogram, weighted by gradient magnitudes, is created. In \figref{fig:sift_descriptor}, we present the situation for a smaller ($8\times8$) window. These histograms form a descriptor vector.

\begin{figure}
    \centering
    \includegraphics[width=0.8\textwidth]{Figures/sift/descriptor.jpg}
    \caption[Extracting the sift-descriptor.]{Building of the sift-descriptor \cite{Lowe2004}.}
    \label{fig:sift_descriptor}
\end{figure}


\section{Speeded Up Robust Features}
Speeded Up Robust Features (SURF) is a local feature transformation algorithm proposed by Herbert Bay, Tinne Tuytelaars, and Luc Van Gool in 2006 \cite{Bay2006}. Compared to SIFT \cite{Lowe1999}, the authors claim the SURF detector and descriptor to be faster and more robust against various image transformations.

\subsection{Key-point Detection}
The SURF key-point detection is based on the determinant of the Hessian matrix. Let us consider an input image $I_{img}(x,y)$ and the scale-space representation of the image
\begin{equation}
    L(x, y,\sigma) =  G(x,y,\sigma)*I_{img}(x,y),
\end{equation}
where $G(x,y,\sigma)$ is the Gaussian kernel defined in \eqref{eq:Gaussian_kernel}.

The Hessian matrix at scale $\sigma$ is defined as
\begin{equation}
    \mathcal{H}(x, y, \sigma) =
    \begin{bmatrix}
        L_{xx}(x, y, \sigma) & L_{xy}(x, y, \sigma)\\
        L_{xy}(x, y, \sigma) & L_{yy}(x, y, \sigma)
    \end{bmatrix},
\end{equation}
where $L_{xx}(x, y, \sigma)$, $L_{xy}(x, y, \sigma)$, and $L_{yy}(x, y, \sigma)$ are the second-order derivatives of the scale-space representation of the image at a point $(x, y)$.

The SURF authors decided to approximate the second order derivatives of Gaussian filter with box filters (shown in \figref{fig:Gauss_box_y}). Because the property of Gaussian filter, that no new structures can appear while going to a lower resolution, has been shown not to apply in 2D case \cite{Koenderink1984}. Box filters allow for the use of integral images, which reduces the computational complexity.

\begin{figure}
    \centering
    \begin{subfigure}{0.24\textwidth}
        \centering
        \includegraphics[width=\textwidth]{Figures/surf/y_gauss.png}
        \label{fig:y_gauss}
    \end{subfigure}
    \begin{subfigure}{0.24\textwidth}
        \centering
        \includegraphics[width=\textwidth]{Figures/surf/y_box.png}
        \label{fig:y_box}
    \end{subfigure}
    \caption[Gaussian second-order derivative in the $y$-direction and its approximation using box filters]{Gaussian second order derivative in the $y$-direction and its approximation using box filters \cite{Bay2006}.}
    \label{fig:Gauss_box_y}
\end{figure}

Let us denote the approximations by $D_{xx}$, $D_{xy}$, and $D_{yy}$. The relative weights in the calculation of determinant of Hessian need to be weighted by $0.9$, which yields
\begin{equation}
    \text{det}(\mathcal{H}) = D_{xx} * D_{yy} - (0.9 * D_{xy})^{2}.
\end{equation}

Due to the use of box filters and integral images, any size of the box filter can be applied to the original image at the same speed directly. Therefore, the scale space is created by the use of up-scaled filters of sizes $9\times9$, $15\times15$, $21\times21$, $27\times27$, etc. For each octave, the difference between filter sizes is doubled (from $6$ to $12$ to $24$).

As the box filter layout remains the same, the corresponding Gaussian filter scales accordingly. The $9\times9$ box filter corresponds to a Gaussian filter with the scale $\sigma = 1.2$. Based on the definition of the process, we can calculate the corresponding Gaussian filter scale for each box filter size.

To select the key-points, non-maximum suppression in the $3\times3\times3$ neighborhood of each point is applied. Afterwards, the maxima of the determinant of the Hessian matrix are interpolated the same way as in the SIFT algorithm (using quadratic Taylor expansion).

\subsection{Key-point Description}
At first, we need to ensure the descriptor's invariance to rotation. For every key-point, we calculate the Haar-wavelet responses in the $x$ and the $y$ direction in a circular neighborhood of $6s$, where $s$ is the Gaussian filter scale at which the key-point was found. Integral images are used for speeding up the process.

The wavelet responses are weighted with a Gaussian($\sigma = 2.5 s$) centered at the key-point. The weighted responses are represented as vectors in a space with the $x$ response being a vector along the $x$-axis, and the $y$ response being a vector along the $y$-axis. All the vectors within a sliding window of $\frac{\pi}{3}$ are summed and the longest of the vector is selected as the key-point orientation.

An example of key-points and their orientation, found by OpenCV SURF implementation, is presented in \figref{fig:surf_example}.
\begin{figure}[ht!]
    \centering
    \includegraphics[width=0.60\textwidth]{Figures/surf/surf_example.jpg}
    \caption[SURF key-points and their orientation from an image of a dog]{SURF key-points and their orientation from an image of a dog.}
    \label{fig:surf_example}
\end{figure}

The descriptor is constructed from a square window the size of $20s$ around a key-point. The window is subsequently rotated along with the key-point orientation. Afterwards, this window is divided into $16$ ($4\times4$) square sub-windows. In each sub-window, the features are calculated using $5\times5$ regularly spaced sample points. From these sample points, we calculate the Haar wavelet responses in the horizontal and the vertical direction, where \say{horizontal} and \say{vertical} are defined in relation to the key-point orientation. The responses are weighted with a Gaussian ($\sigma = 3.3s$) centered at the key-point.

Let us call the horizontal responses $d_x$ and the vertical responses $d_y$. Over each sub-window, we denote the sum of the responses $\sum d_x$ and $\sum d_y$, and the sum of absolute values of the responses $\sum |d_x|$ and $\sum |d_y|$. The behavior of these values for different image patterns can be seen in \figref{fig:surf_descriptor}. Combining $\sum d_x$, $\sum d_y$, $\sum |d_x|$, and $\sum |d_y|$ for each of the $16$ sub-window into a vector, we get a descriptor of length $64$.

\begin{figure}
    \centering
    \includegraphics[width=0.8\textwidth]{Figures/surf/surf_descriptor.png}
    \caption[Behaviour of SURF descriptor for different image patterns]{Behaviour of SURF descriptor for different image patterns \cite{Bay2006}.}
    \label{fig:surf_descriptor}
\end{figure}


\section{Oriented FAST and Rotated BRIEF}
Oriented FAST and Rotated BRIEF (ORB) is a local feature extractor proposed by Ethan Rublee, Vincent Rabaud, Kurt Konolige, and Gary Bradski in 2011 \cite{Rublee2011}.

The algorithm is based on the FAST (Features from Accelerated Segment Test) corner detector \cite{Rosten2006} and the BRIEF (Binary Robust Independent Elementary Features) descriptor \cite{Calonder2010}.

\subsection{Key-point Detection}
The ORB algorithm uses the FAST-9 variant of the FAST corner detector. This detector compares each pixel intensity (denoted as $I_p$) in an image with the intensities of pixels in a circle of radius $9$ around the pixel.

Let $n\in\mathbb{N}$ and the threshold $t\in\mathbb{R}^{+}$ be given. Let $S$ be a set of pixels in a circle of radius $9$ around the examined pixel. Let us denote $I_x$ as the intensity of a pixel $x$. The examined pixel is selected as a corner if there exists a set $S_n \in S$ of $n$ contiguous pixels, where $\forall x \in S_n: I_x + t < I_p$, or $\forall x \in S_n: I_x - t > I_p$.

The selected corners, i.e., key-points are then ordered according to a Harris corner measure \cite{Harris1988}. For $N$ key-points, the threshold is selected low enough to get more than $N$ key-points. The best $N$ key-points (according to the Harris corner measure) are then selected.

To find multi-scale features, the scale pyramid of an image is generated and key-points are located at each level in the pyramid.

\subsection{Key-point Description}
To ensure the descriptor's invariance to rotation, we need to determine the key-point orientation. To achieve this goal, the intensity centroid \cite{Rosin1999} is used. The intensity centroid expects the intensity of a key-point to be offset from its center. The vector from the center to the centroid is used for the key-point orientation.

Given image $I(x,y)$, a moment of a patch is defined as
\begin{equation}
    m_{pq}\coloneqq\sum_{x,y} x^p y^q I(x,y),
\end{equation}
and the centroid is located as
\begin{equation}
    C =
    \begin{pmatrix}
        \frac{m_{10}}{m_{00}} \text{, } \frac{m_{01}}{m_{00}}
    \end{pmatrix},
\end{equation}
from which we can obtain the key-point orientation
\begin{equation}
    \theta = \text{atan}2(m_{01},m_{10}),
\end{equation}
where atan$2$ is the quadrant-aware version of arctan.

An example of key-points and their orientation detected by OpenCV ORB implementation, can be seen in \figref{fig:orb_example}.
\begin{figure}[ht!]
    \centering
    \includegraphics[width=0.65\textwidth]{Figures/orb/orb_example.jpg}
    \caption[ORB key-points and their orientation from an image of a cat]{ORB key-points and their orientation from an image of a cat.}
    \label{fig:orb_example}
\end{figure}

As an ORB descriptor, a variation on the BRIEF descriptor is used. The BRIEF descriptor is a vector of binary values. This allows for fast matching using a Hamming distance.

The descriptor values are computed by comparing random pairs of pixel intensities in a patch. The binary vector is based on the test defined as
\begin{equation}
    \tau(\boldsymbol{p};x,y)=
    \begin{cases*}
        1 & if $\boldsymbol{p}(x) < \boldsymbol{p}(y)$, \\
        0 & otherwise,
    \end{cases*}
\end{equation}
where $\boldsymbol{p}(x)$ is the pixel intensity of a patch $\boldsymbol{p}$ at a point $x$. The pairs of points $x$ and $y$ are from a random predetermined set
\begin{equation}
    \boldsymbol{S} =
    \begin{pmatrix}
        x_1 \dots x_n \\
        y_1 \dots y_n
    \end{pmatrix},
\end{equation}
where $n$ is the size of our descriptor. The BRIEF descriptor is defined as a vector of $n$ binary tests
\begin{equation}
    f_n(\boldsymbol{p}) \coloneqq \sum_{i=1}^{n} 2^{i-1}\tau(\boldsymbol{p};x_i, y_i).
\end{equation}

The steered version of the BRIEF descriptor, according to the orientation $\theta$ of the key-point, can be created by using a corresponding rotation matrix $\boldsymbol{R}_\theta$:
\begin{equation}
    \boldsymbol{S_\theta} = \boldsymbol{R}_\theta \boldsymbol{S}.
\end{equation}
The sets of pairs are precomputed in a lookup table for discretized $\theta$ in increments of $\frac{2\pi}{30}$ to improve speed. In the end, the steered BRIEF descriptor becomes
\begin{equation}
    g_n(\boldsymbol{p}, \theta) \coloneqq f_n(\boldsymbol{p}) | (x_i, y_i) \in \boldsymbol{S}_\theta.
\end{equation}


\section{Bag of Visual Words}
Local feature extractors provide us with varied number of descriptors for each image. However, for classification, we require each image to be represented by a single vector. This can be achieved using a Bag of Words technique.

Bag of Words (BoW) is a technique, which represents each sample in a dataset as a multiset of its words. In general, these words do not have to be literal words but can be any categorization of information provided by the sample.

The first step of BoW is the dictionary generation. The dictionary is a set of $k \in \mathbb{N}$ categories of words called bins. Provided the dictionary, the BoW assigns each word to a bin. Finally, for each sample in the dataset, a histogram of frequencies of words belonging to each bin is calculated.

As an example, we explain the technique with text documents. With text documents, the dictionary consists of every unique word in all documents. The words of each document are assigned to a bin consisting of said words.

For example, in a dictionary consisting of bins [ \say{Peter}, \say{run}, \say{talked}, \say{to} ], a sample sentence \say{Peter talked to Peter} would produce the following histogram:
\begin{equation}
    [2, 0, 1, 1].
\end{equation}

In our case, we consider the descriptors found in an image as words. Since we are dealing with visual information, let us call these words \say{visual words} and designate the BoW technique \say{Bag of Visual Words} (BoVW).

We generate the categories using a $k$-means algorithm on all the visual words in the training dataset. Afterwards, we assign the visual words to the closest (considering euclidean distance) category and compute the histogram of frequencies.

\subsection{$k$-means}
The $k$-means algorithm is an unsupervised learning technique \cite{macqueen1967}. Let $X=\{ x_1, x_2, \dots, x_n \}$ be a dataset of $n$ data-points. The goal is to assign the data points into $k$ clusters $S = \{ S_1, S_2, \dots, S_k \}$, such that
\begin{equation}
    \argmin_S \sum_{i=1}^k \sum_{x\in S_i} \norm{x-c_i}^2,
\end{equation}
where \(c_1, c_2, ..., c_k \) are the centroids of corresponding clusters \(S_1, S_2, ..., S_k \). The centroids are determined as means of data points belonging to each cluster.

The algorithm starts by selecting $k$ random data points as the centroids. The iterative process is based on the alternation between the following two steps until a termination condition is met:
\begin{description}
    \item{\textbf{Assignment step:}} Each data point is assigned to the cluster, such that the Euclidean distance between the centroids of the clusters and the data point is the smallest one.
    \item{\textbf{Update step:}} For each cluster, a new centroid is computed as the mean value of data points belonging to this cluster.
\end{description}

The termination condition is satisfied, when the ratio of the samples, for which the assigned cluster changes, to the total amount of samples, is smaller than a specified threshold. The steps of the algorithm can be seen in \figref{fig:k_means_alg}.
\begin{figure}[ht]
    \centering
    \begin{subfigure}[t]{0.22\textwidth}
        \includegraphics[width=\textwidth]{Figures/k-means/k-means_inicial_step.png}
        \caption{At first, random data points are selected as the initial $k$ centroids.}
        \label{fig:k-means-alg:inicial_step}
    \end{subfigure}\hfill
    \begin{subfigure}[t]{0.22\textwidth}
        \includegraphics[width=\textwidth]{Figures/k-means/k-means_assignment_step.png}
        \caption{By choosing the nearest centroid, each data point is assigned to a corresponding cluster.}
        \label{fig:k-means-alg:assignment_step}
    \end{subfigure}\hfill
    \begin{subfigure}[t]{0.22\textwidth}
        \includegraphics[width=\textwidth]{Figures/k-means/k-means_update_step.png}
        \caption{New centroids are calculated as mean values of data points in the cluster.}
        \label{fig:k-means-alg:update_step}
    \end{subfigure}\hfill
    \begin{subfigure}[t]{0.22\textwidth}
        \includegraphics[width=\textwidth]{Figures/k-means/k-means_assignment_step_2.png}
        \caption{Steps in (b) and (c) are repeated until the terminate condition is met.}
        \label{fig:k-means-alg:assignment_step_2}
    \end{subfigure}\hfill
    \caption[Lloyd's algorithm demonstration.]{Lloyd algorithm demonstration \cite{Wikikmeans}.}
    \label{fig:k_means_alg}
\end{figure}
The challenge is the determination of the number of clusters $k$, and consequently the size of the visual dictionary. In our experiments, we examine different settings for $k$ to find a clustering, which best represents our data.


\section{PCA}
PCA (Principal Component Analysis) is a global feature extractor. It is used to reduce the dimension of data while preserving as much of the data variance as possible.

Let $x_t \in \mathbb{R}^n$ be a data point, where $t=1,\dots,T$ and $T$ is the amount of data points. We want to project $x_t$ into a $y_t \in \mathbb{R}^k$, where $k < n$. We want to find the optimal parameters $P$ of the parametric reduction mapping $\psi_P: \mathbb{R}^n \rightarrow \mathbb{R}^k$ and the parametric reconstruction mapping $\psi_P^{-1}: \mathbb{R}^k \rightarrow \mathbb{R}^n$. The optimal parameters $P^{*}$ minimize the reconstruction error, i.e.,
\begin{equation}
    P^{*} \coloneqq \argmin_{P \in \rho} \sum_{t=1}^T \norm{x_t - \psi_P^{-1}(\psi_P(x_t))},
    \label{eq:mapping_optimization}
\end{equation}
where $\psi$ denotes the set of feasible parameters and $\norm{x_t - \psi_P^{-1}(\psi_P(x_t))}$ is the reconstruction error of a data-point.

PCA uses the mappings
\begin{equation}
    \psi_{[Q,b]}^{-1}(y) \coloneqq Qy+b, \psi_{[Q,b]}(x) \coloneqq Q^{\top}(x-b),
    \label{eq:pca_mappings}
\end{equation}
with parameters $b \in \mathbb{R}^n$ and $Q \in \mathbb{R}^{n,k}$ with orthonormal columns, i.e.,
\begin{equation}
    Q^{\top}Q = I \in \mathbb{R}^{k,k}.
    \label{eq:pca_identity}
\end{equation}

Substituting \eqref{eq:pca_mappings}, \eqref{eq:pca_identity} into the optimization problem \eqref{eq:mapping_optimization} and using the square Euclidean norm for measuring the reconstruction error, we get the optimization problem
\begin{equation}
    [Q^{*}, b^{*}] \coloneqq \argmin_{Q,b} \sum_{t=1}^T \norm{x_t - (QQ^{\top}(x_t-b) + b)}^2_2 \text{s.t.} Q^{\top}Q=I.
    \label{eq:pca_optimization}
\end{equation}
The objective function $f(Q,b)$ of \eqref{eq:pca_optimization} can be rewritten to the form
\begin{equation}
    f(Q,b)=\sum_{t=1}^T (x_t^{\top}x_t - 2 x_t^{\top}b + b^{\top}b - x_t^{\top}QQ^{\top}x_t + 2x_t^{\top}QQ^{\top}b - b^{\top}QQ^{\top}b),
    \label{eq:pca_obj_fun}
\end{equation}
and the optimality condition of $b^{*}$ can be formulated as
\begin{equation}
    \nabla_bf(Q,b) = \sum_{t=1}^T(-2x_t+2b+2QQ^{\top}xt-2QQ^{\top}b)=0,
\end{equation}
which is equivalent to
\begin{equation}
    (I-QQ^{\top})\sum_{t=1}^T(b-x_t)=0.
    \label{eq:pca_opt_b}
\end{equation}
As $k<n$, $Q \in \mathbb{R}^{n,k}$ is not a full column rank and $QQ^{\top} \neq = I$. The unique solution of \eqref{eq:pca_opt_b} is
\begin{equation}
    b^{*} = \frac{1}{T}\sum_{t=1}^T x_t.
\end{equation}
Moreover, the Hessian matrix
\begin{equation}
    \nabla_{b,b}^2 f(Q,b)=2(I-QQ^{\top}),
    \label{eq:pca_sol_b}
\end{equation}
is symmetric positive definitive matrix, therefore the objective function \eqref{eq:pca_obj_fun} is strictly convex (in the variable $b$) and \eqref{eq:pca_sol_b} is unique minimizer.

Let us denote the shifted data by
\begin{equation}
    \hat{x_t} \coloneqq x_t - b^* , t=1, \dots, T
\end{equation}
and write the objective function of \eqref{eq:pca_obj_fun} as
\begin{equation}
    f(Q,b^*) = \sum_{t=1}^T \norm{\hat{x_t} - QQ^{\top}\hat{x_t}}_2^2 = \sum_{t=1}^T (\hat{x_t}^{\top}\hat{x_t} - \hat{x_t}^{\top} QQ^{\top} \hat{x_t}).
    \label{eq:pca_obj_fun_Q}
\end{equation}
We can simplify \eqref{eq:pca_obj_fun_Q} using properties of the matrix trace:
\begin{equation}
    f(Q,b^*) = \sum_{t=1}^T \text{trace} (\hat{x_t}^{\top}\hat{x_t}) - \sum_{t=1}^T \text{trace} (\hat{x_t}^{\top} QQ^{\top} \hat{x_t})
    = \text{trace}(\text{cov}(x)) - \text{trace}(Q^{\top} \text{cov}(x) Q ),
\end{equation}
where cov($x$) is the covariance matrix of $x$ defined as
\begin{equation}
    \text{cov}(x) \coloneqq \sum_{t=1}^T(x_t-b^*)(x_t-b^*)^{\top} = \sum_{t=1}^T\hat{x_t}\hat{x_t}^{\top}.
\end{equation}
The argument of the minimum is independent of constants in the objective function. Therefore, we can reformulate the optimization problem \eqref{eq:pca_optimization} (in terms of variable $Q$) as
\begin{equation}
    \begin{split}
        [Q^*] &= \argmin_{Q^{\top}Q=I} f(Q,b^*) = \argmin_{Q^{\top}Q=I} -\text{trace}(Q^{\top}\text{cov}(x)Q) \\&= \argmax_{Q^{\top}Q=I} \text{trace}(Q^{\top}\text{cov}(x)Q) \\&= \argmax_{\forall j:q_j^{\top}q_j=1} \sum_{j=1}^k q_j^{\top} \text{cov}(x) q_j,
    \end{split}
    \label{eq:pca_optimization_final}
\end{equation}
where $q_j , j=1,\dots,k$ denote the (orthonormal) columns of the matrix $Q \in \mathbb{R}^{n,k}$. The Lagrange function corresponding to the problem \eqref{eq:pca_optimization_final} is given by
\begin{equation}
    L(Q, \lambda) \coloneqq \sum_{j=1}^k q_j^{\top} \text{cov}(x)q_j - \sum_{j=1}^k \lambda_j (q_j^{\top} q_j - 1),
\end{equation}
where $\lambda \in \mathbb{R}^K$ denotes Lagrange multiplier corresponding to equality constraints. The first Karush-Kuhn-Tucker conditions can be derived as
\begin{equation}
    \nabla_{q_j}L(Q,\lambda) = 2 \text{cov}(x)q_j-2\lambda_jq_j=0, j=1,\dots,k,
\end{equation}
which are the eigenvalue equations
\begin{equation}
    \text{cov}(x)q_j=\lambda_j q_j, j=1,\dots,k.
    \label{eq:pca_eig}
\end{equation}
If we substitute \eqref{eq:pca_eig} into objective function \eqref{eq:pca_optimization_final}, we get
\begin{equation}
    \sum_{j=1}^k q_j^{\top} \text{cov}(x)q_j = \sum_{j=1}^k \lambda_j q_j^{\top}q_j = \sum_{j=1}^k \lambda_j.
\end{equation}
As the problem \eqref{eq:pca_optimization_final} is maximization problem, the optimal $Q^*$ consists of (orthonormal) eigenvectors which correspond to $k$ largest eigenvalues.




\chapter{Classifiers}
For data classification we are using two classification techniques, the Bayes model and SVM. The goal of these techniques is to find a mapping from a feature space into a space of labels.

\section{Bayesian Model}
This classifier is suitable for classifying data represented by a stochastic vector. The BoVW data can be easily transformed into a such vector. Instead of each component of the BoVW vector representing the number of key-points in the respective category, the component in our new vector represents the probability of key-points belonging to the respective category.

Let us denote the stochastic data vector $\Pi_{xt} \in \mathbb{R}^{K_x}, t=1,\dots,T$, where $T$ is the number of samples and $K_x$ is the size of BoVW. Let the vector $\Pi_{yt}\in \mathbb{R}^{K_y}$ be a vector of probabilities, with which $\Pi_{xt}$ belongs to each category, and $K_y$ the number of categories.

Given a stochastic data vector $\Pi_x$, we can describe the transformation $\mathbb{R}^{K_x} \rightarrow \mathbb{R}^{K_y}$ using a matrix $\Delta \in \mathbb{R}^{K_y, K_x}$:
\begin{equation}
    \Delta =
    \begin{bmatrix}
        P(y_t = y_1 | x_t = x_1) & P(y_t = y_1 | x_t = x_2) & \dots & P(y_t = y_1 | x_t = x_{K_x})\\
        P(y_t = y_2 | x_t = x_1) & P(y_t = y_2 | x_t = x_2) & \dots & P(y_t = y_2 | x_t = x_{K_x})\\
        \vdots & \ddots\\
        P(y_t = y_{K_y} | x_t = x_1) & P(y_t = y_{K_y} | x_t = x_2) & \dots & P(y_t = y_{K_y} | x_t = x_{K_x})\\
    \end{bmatrix},
\end{equation}
where $\Pi_x^n$ is the $n$-th element of $\Pi_x$, similar to $\Pi_y^n$, and the matrix $\Delta$ is a left stochastic matrix.

The search for the optimal $\Delta^{*}$ can be written as
\begin{equation}
    \Delta^* = \argmin_{\Delta \in \Omega_{\Delta}} \sum_{t=1}^T \text{dist}(\Pi_{yt}, \Delta\Pi_{xt}),
\end{equation}
where $\Omega_{\Delta}$ is a set of left stochastic matrices. The dist($\Pi_{yt}, \Delta\Pi_{xt}$) is calculated as Kullback-Leiber divergence\cite{Kullback1951}:
\begin{equation}
    \text{dist}(\Pi_{yt}, \Delta\Pi_{xt}) = - \sum_{i=1}^{K_y} \Pi_{yt}^i \ln\frac{(\Delta\Pi_{xt})_i}{\Pi_{yt}^i} = - \sum_{i=1}^{K_y} \Pi_{yt}^i (\ln (\Delta\Pi_{xt})_i - \ln \Pi_{yt}^i).
\end{equation}
For the optimization, the term $\ln \Pi_{yt}^i$ is constant, therefore it can be ignored:
\begin{equation}
    \text{dist}(\Pi_{yt}, \Delta\Pi_{xt}) \propto - \sum_{i=1}^{K_y} \Pi_{yt}^i \ln (\Delta\Pi_{xt})_i.
\end{equation}
This problem is hard to minimize analytically. However, $\Delta\Pi_{xt}$ is a convex combination, and therefore $-\ln(\Delta\Pi_{xt})$ is a convex function. Thus Jensen's inequality can be used:
\begin{equation}
    - \sum_{i=1}^{K_y} \Pi_{yt}^i \ln (\Delta\Pi_{xt})_i \leq - \sum_{i=1}^{K_y} \Pi_{yt}^i ( \sum_{j=1}^{K_x} \Pi_{xt}^j \ln (\Delta_{ij}) ) = - \sum_{i=1}^{K_y} \sum_{j=1}^{K_x} \Pi_{yt}^i \Pi_{xt}^j \ln \Delta_{ij}.
\end{equation}
From this, we get an optimization problem
\begin{equation}
    \Delta^* = \argmin_{\Delta \in \Omega_{\Delta}} - \sum_{t=1}^T \sum_{i=1}^{K_y} \sum_{j=1}^{K_x} \Pi_{yt}^i \Pi_{xt}^j \ln \Delta_{ij},
\end{equation}
where
\begin{equation}
    \Omega_{\Delta} = \{ \Delta \in [0, 1], \forall j \in \{ 1, 2, \dots, K_x \} : \sum_{i=1}^{K_y} \Delta_{ij} = 1 \},
\end{equation}
which is a feasible set of left stochastic matrices. The problem can be solved analytically.

Let $\Delta$ be the optimal stochastic matrix. There exist $\lambda_j \in \mathbb{R}^{K_x}$ such that
\begin{equation}
    L(\Delta, \lambda) = - \sum_{t=1}^T \sum_{i=1}^{K_y} \sum_{j=1}^{K_x} \Pi_{yt}^i \Pi_{xt}^j \ln \Delta_{ij} + \sum_{j=1}^{K_x} \lambda_j (\sum_{i=1}^{K_y} \Delta_{ij} - 1)
\end{equation}
is the Lagrange function. The Karush-Kuhn Tucker conditions are
\begin{equation}
    \nabla_{\Delta_{\hat{i}\hat{j}}} L(\Delta_{\hat{i}\hat{j}}, \lambda) = - \frac{1}{\Delta_{\hat{i}\hat{j}}} \sum_{t=1}^{T} \Pi_{yt}^{\hat{i}} \Pi_{xt}^{\hat{j}} + \lambda_{\hat{j}} = 0,
    \label{eq:bayes_first_kkt}
\end{equation}
and
\begin{equation}
    \nabla_{\Delta_{\hat{j}}} L = \sum_{i=1}^{K_y} \Delta_{i\hat{j}} - 1 = 0.
    \label{eq:bayes_second_kkt}
\end{equation}
From \eqref{eq:bayes_first_kkt} and \eqref{eq:bayes_second_kkt}, we get the optimal $\Delta^*$ with components
\begin{equation}
    \Delta_{\hat{i}\hat{j}}^{*} = \frac{\sum_{t=1}^{T} \Pi_{yt}^{\hat{i}} \Pi_{xt}^{\hat{j}}}{\sum_{i=1}^{K_y} \sum_{t=1}^{T} \Pi_{yt}^{i} \Pi_{xt}^{\hat{j}},}.
\end{equation}

Another approach to finding the optimal $\Delta^{*}$ is optimizing the problem
\begin{equation}
    \Delta^* = \argmin_{\Delta \in \Omega_{\Delta}} - \sum_{t=1}^T \sum_{i=1}^{K_y} \Pi_{yt}^i \ln (\Delta\Pi_{xt})_i,
\end{equation}
without using the Jensen inequality. Since the feasible set $\Omega_{\Delta}$ is a closed convex set, the problem can be solved numerically using the Spectral Projected Gradient method\cite{birgin2000}.

We use both approaches (the analytical solution using Jensen inequality and the numerical solution) in our experiments.


\section{SVM}
The SVM (Support Vector Machine) is a supervised learning classifier originally designed for binary classifications. It was introduced by Vladimir N. Vapnik and Alexey Ya. Chervonenkis in 1963 \cite{Cortes1995}.

Let \( T := \{(\boldsymbol{x_1}, y_1),(\boldsymbol{x_2}, y_2),...,(\boldsymbol{x_n}, y_n)\} \),
be the training dataset, where $n$ is the number of the samples, \( \boldsymbol{x_i} \in \mathbb{R}^m \), \( i \in \{1,2,\dots,n\} \)
is the sample and \( y_i \in \{-1, 1\} \) is the label related to the sample \( \boldsymbol{x_i} \). The classification model is represented by the means of the hyperplane \( H \), defined such that:
\begin{equation}
    H: \boldsymbol{\omega}^T\boldsymbol{x}-\widetilde{b}=0,
    \label{eq:svm:hyperplane}
\end{equation}
where \( \omega \) is the normalized normal vector of the hyperplane \( H \), and
\begin{equation}
    \widetilde{b} = \frac{b}{\norm{\omega}}
    \label{eq:svm:offset}
\end{equation}
is the bias from the origin.

First, we consider linearly separable training data. The two classes of data are distinguished by two hyperplanes so that the distance between them is maximized. The region bound by these hyperplanes is called the margin. The hyperplane that lies halfway between them is called the maximum-margin hyperplane. These hyperplanes are described by the following equation:
\begin{equation}
    \boldsymbol{w}^T\boldsymbol{x}-\widetilde{b}=\pm1.
    \label{eq:svm_hyperplanes}
\end{equation}
Sample $x_i, i=1,\dots,n$ belongs to the positive class, i.e. $y_i = 1, i=1,\dots,n$, when
\begin{equation}
    \boldsymbol{w}^T\boldsymbol{x_i}-\widetilde{b}\geq1,
    \label{eq:svm_positive_data}
\end{equation}
and the negative class, i.e. $y_i = -1, i=1,\dots,n$, when
\begin{equation}
    \boldsymbol{w}^T\boldsymbol{x_i}-\widetilde{b}\leq-1.
    \label{eq:svm_negative_data}
\end{equation}
The properties \eqref{eq:svm_positive_data} and \eqref{eq:svm_negative_data} can be combined into a single equation
\begin{equation}
    y_i(\boldsymbol{w}^T\boldsymbol{x}-\widetilde{b})\geq1.
\end{equation}

\begin{figure}[ht!]
    \centering
    \includegraphics[width=0.6\textwidth]{Figures/svm/hyperplane_svm.pdf}
    \caption[Example of the SVM model.]{Example of the SVM model. \cite{Kruzik2018}}
    \label{fig:svm-margin}
\end{figure}

From the \figref{fig:svm-margin}, we can see, the distance between hyperplanes \eqref{eq:svm_hyperplanes} is \( \frac{2}{\norm{\boldsymbol{w}}} \). As we want to maximize this distance, we need to minimize $\norm{\boldsymbol{w}}$. This leads to an optimization problem
\begin{equation}
    \argmin_{\boldsymbol{w},b} \|\boldsymbol{w}\| \text{ s.t. }
    \begin{cases}
        y_i(\boldsymbol{w}^T\boldsymbol{x}_i-b)\geq1,\\
        i=1, ...,n,
    \end{cases}
\end{equation}
which can be reformulated as the Quadratic Programming problem
\begin{equation}
    \argmin_{\boldsymbol{w},b} \frac{1}{2}\|\boldsymbol{w}\|^2 \text{ s.t. }
    \begin{cases}
        y_i(\boldsymbol{w}^T\boldsymbol{x}_i-b)\geq1,\\
        i=1, ...,n.
    \end{cases}
    \label{eq:svm_hard_margin_minimization}
\end{equation}

Because the training data is nearly never linearly separable, the soft margin version of the SVM was proposed by Vladimir N. Vapnik and Corinna Cortes \cite{Cortes1995} in 1995. It exploits an additional function called the hinge loss function:
\begin{equation}
    \xi_i = \max(0, 1-y_i(\boldsymbol{w}^T\boldsymbol{x}_i-b)).
    \label{eq:svm_hinge_loss}
\end{equation}

The hinge loss function \eqref{eq:svm_hinge_loss} equals $0$ for a sample on the correct side of the corresponding hyperplane \eqref{eq:svm_hyperplanes}. However, for a sample on the wrong side of the corresponding hyperplane \eqref{eq:svm_hyperplanes}, the value of the function is proportional to the distance from the hyperplane.

If we add the hinge loss function \eqref{eq:svm_hinge_loss} to optimization problem \eqref{eq:svm_hard_margin_minimization}, we get a soft margin SVM optimization problem
\begin{equation}
    \argmin_{\boldsymbol{w},b,\xi_i} \frac{1}{2}\|\boldsymbol{w}\|^2 + C\sum_{i = 1}^n\xi_i \text{ s.t. }
    \begin{cases}
        y_i(\boldsymbol{w}^T\boldsymbol{x}_i-b)\geq1 - \xi_i,\\
        \xi_i \geq 0 , i=1, ...,n,
    \end{cases}
    \label{eq:svm:soft_margin_minimization}
\end{equation}
where \( C \) is a penalty for the misclassification error. The formulation \eqref{eq:svm:soft_margin_minimization} is called the $l1$-loss $l2$-regularized SVM. The primal formulation \eqref{eq:svm:soft_margin_minimization} can be modified using the Lagrange duality with Lagrange multipliers $\boldsymbol{\alpha} = [\alpha_1, \alpha_2, ..., \alpha_n]^{\top}$, $\boldsymbol{\beta} = [\beta_1, \beta_2, ..., \beta_n]^{\top}$. Exploiting Karush-Kuhn-Tucker conditions, we obtain the dual formulation
\begin{equation}
    \argmin_{\boldsymbol{\alpha}} \frac{1}{2} \boldsymbol{\alpha}^T\boldsymbol{Y}^T\boldsymbol{K}\boldsymbol{Y}\boldsymbol{\alpha} - \boldsymbol{\alpha}^T\boldsymbol{e} \text{ s.t. } 
    \begin{cases}
        \boldsymbol{o} \leq \boldsymbol{\alpha} \leq C\boldsymbol{e},\\
        \boldsymbol{B_e}\boldsymbol{\alpha}=0,
    \end{cases}
    \label{eq:svm:soft_margin_dual}
\end{equation}
where \( \boldsymbol{e} = [1,1, \dots,1]^{\top} \), \( \boldsymbol{o} = [0,0, \dots,0]^{\top} \), \( \boldsymbol{X} = [\boldsymbol{x_1},\boldsymbol{x_2}, \dots,\boldsymbol{x_n}] \), \( \boldsymbol{y} = [y_1,y_2, \dots,y_n]^{\top} \), \( Y = \text{diag}(\boldsymbol{y}) \), \( \boldsymbol{B_e} = [\boldsymbol{y}^T] \) and \( \boldsymbol{K}\coloneqq\boldsymbol{X}^T\boldsymbol{X} \) is the Gram matrix which is symmetric positive semi-definite (SPSD)\cite{Aeta2018}. The Hessian matrix in \eqref{eq:svm:soft_margin_dual}
\begin{equation}
    \boldsymbol{H} \coloneqq \boldsymbol{Y}^T\boldsymbol{X}^T\boldsymbol{X}\boldsymbol{Y}
    \label{eq:svm:hessian}
\end{equation}
is also an SPSD matrix.

To recover the normal vector, the formula
\begin{equation}
    \boldsymbol{w}=\boldsymbol{X}\boldsymbol{Y}\boldsymbol{\alpha}.
    \label{eq:svm:dual_to_primal_w}
\end{equation}
is used. The bias $b$ can be recovered as
\begin{equation}
    b=\boldsymbol{w}\cdot \boldsymbol{\Bar{x}} - y_i,
    \label{eq:svm:dual_to_primal_b}
\end{equation}
where $\boldsymbol{\Bar{x}}$ is the mean of all support vectors.

Instead of the linear sum of the loss functions $\xi_i$ in \eqref{eq:svm:soft_margin_minimization}, we can use the square sum of the loss functions in the objective function
\begin{equation}
    \argmin_{\boldsymbol{w},b,\xi_i} \frac{1}{2}\|\boldsymbol{w}\|^2 + \frac{C}{2}\sum_{i = 1}^n\xi_i^2 \text{ s.t. } 
    \begin{cases}
        y_i(\boldsymbol{w}^T\boldsymbol{x}_i-b)\geq1 - \xi_i,\\
        i=1, ...,n.
    \end{cases}
    \label{eq:svm:soft_margin_minimization_l2}
\end{equation}
The problem \eqref{eq:svm:soft_margin_minimization_l2} is called $l2$-loss $l2$-regularized SVM. The dual formulation can again be obtained by using the Lagrange duality:
\begin{equation}
    \argmin_{\boldsymbol{\alpha}} \frac{1}{2} \boldsymbol{\alpha}^T(\boldsymbol{H}+C^{-1}\boldsymbol{I})\boldsymbol{\alpha} - \boldsymbol{\alpha}^Te \text{ s.t. } 
    \begin{cases}
        \boldsymbol{o} \leq \boldsymbol{\alpha},\\
        \boldsymbol{B_e}\boldsymbol{\alpha}=0.
    \end{cases}
    \label{eq:svm:soft_margin_dual-l2}
\end{equation}

The Hessian matrix $\boldsymbol{H}$, regularized by the matrix $C^{-1}\boldsymbol{I}$ is symmetric positive definite, therefore, this optimization problem should be more stable than the $l1$-loss $l2$-regularized SVM problem.

\subsection{Hyperparameter optimization}
Hyperparameter optimization is the process of selecting good parameters for a classifier. In SVM, we need to find a good penalty $C$. One approach to hyperparameter optimization is a Grid search.

Grid search is an exhaustive searching through a user-specified subset of parameters. The classifier is trained with each combination of the subset. The combination, which yields the best result is then selected. By the best result, we mean the highest score of a user-specified metric.

To test the hyperparameter selection, we need new, independent data from the data used in a training step. This is accomplished by splitting the data into subsets. The subsets are selected using a stratified cross-validation technique.

\subsection{Cross-validation}
Cross-validation is an approach to split a dataset into a training and validation subset. For each set of hyperparameters, a dataset is split into $k$ subsets (folds) of an equal size. One of the subsets is kept as a validation set, while the model is trained on the other $k-1$ folds. The process is repeated $k$ times, selecting each fold as a validation subset. The results of all $k$ runs are averaged for a final score. This process can be seen in \figref{fig:k-fold}.
\begin{figure}
    \centering
    \includegraphics[width=0.7\textwidth]{Figures/svm/k-fold.jpg}
    \caption[$k$-fold CV splits data into $k$ folds and trains the model $k$ times always leaving out a different fold as a validation set.]{$k$-fold CV splits data into $k$ folds and trains the model $k$ times always leaving out a different fold as a validation set. \cite{crossval}}
    \label{fig:k-fold}
\end{figure}

As a randomly selected fold might not represent the classes in the same ratio as the whole set, stratified cross-validation is used. The random folds are selected in a way, where the ratio of the classes is roughly equal to the ratio of classes in the whole dataset.


\section{Metrics}
To assess the quality of the classification model, we need to analyze several metrics. Many of such metrics are derived from a confusion matrix (see \tabref{tab:confusion_matrix}).
\begin{table}[ht]
    \centering
    \begin{tabular}{|c|c|c|}
        \hline
        & Predicted negative & Predicted positive \\ 
        \hline 
        Actual negative & True negatives ($TN$) & False positives ($FP$) \\ 
        \hline
        Actual positive & False negatives ($FN$) & True positives ($TP$)  \\ 
        \hline
    \end{tabular}
    \caption{Confusion Matrix}
    \label{tab:confusion_matrix}
\end{table}
The confusion matrix is generated by counting the testing data, which are supposed to belong to a negative class (Actual negative) or a positive class (Actual positive), and their predicted class (Predicted negative/Predicted positive).

We are considering the following metrics in our experiments:
\begin{description}
    \item{\textbf{Accuracy:}} How often is the classifier correct overall. This metric is represented in percentages.
    \begin{equation}
        \frac{TN+TP}{TN+FP+FN+TP}
        \label{eq:svm:accuracy}
    \end{equation}
    \item{\textbf{Precision:}} How often is the classifier correct when it predicts positive.
    \begin{equation}
        \frac{TP}{TP+FP}
        \label{eq:svm:precision}
    \end{equation}
    \item{\textbf{Sensitivity (Recall):}} How often is the classifier correct when it is actually positive.
    \begin{equation}
        \frac{TP}{TP+FN}
        \label{eq:svm:sensitivity}
    \end{equation}
    \item{\textbf{\( F_1 \) Score:}} A harmonic mean of precision and sensitivity.
    \begin{equation}
        2\times\frac{\text{precision}\times\text{recall}}{\text{precision}+\text{recall}}
        \label{eq:svm:F1}
    \end{equation}
\end{description}

Generally, for each of these metrics, the higher value we get, the better the classification model.



\chapter{Datasets}
To test our extraction-classification pipeline, we use four progressively more complex datasets. We start by classifying simple 2D shapes, then move on to simple 3D shapes, and last, we attempt to classify a dataset of real photographs of cats and dogs.

\section{2D Shapes Dataset}
For the simplest dataset, we use a four shapes dataset \cite{kaggleFourShapes}. The dataset contains $16000$ images of four shapes: square, star, circle, and triangle. The dataset was created from poster board cutouts of the shapes, painted green. While rotating, each shape was recorded using a Garmin Virb 1080p action camera for two minutes. Shapes were then cropped out from the frames of the video and resized to $200\times200$ pixels. The green color of the cutouts in frames is then changed to a pure black color, while the rest of the image is changed to a pure white color. An example of the data can be seen in \figref{fig:four_shapes}.

As we are doing a binary classification, we use only the pair of a circle and a star.
\begin{figure}[ht]
    \centering
    \begin{subfigure}[t]{0.2\textwidth}
        \includegraphics[width=\textwidth]{Figures/datasets/circle.png}
        \caption{Circle}
    \end{subfigure}
    \begin{subfigure}[t]{0.2\textwidth}
        \includegraphics[width=\textwidth]{Figures/datasets/star.png}
        \caption{Star}
    \end{subfigure}
    \caption[Photographs of a circle and a star from the Four Shapes dataset]{Photographs of a circle and a star from the Four Shapes dataset \cite{kaggleFourShapes}.}
    \label{fig:four_shapes}
\end{figure}

\section{3D Shapes Dataset}
The 3D Shapes dataset contains $20$ images of a box and $20$ images of a sphere from different points of view. The shapes are colored red, while the background is colored white. The objects are illuminated by a single light source, providing a nice and sharp shadow and shading of the each shape. The dataset was generated by Ing. Lukáš Pospíšil, Ph.D. in POV-Ray \cite{povray}. Each of the images is a $300\times200$ pixels. An example of each shape can be seen in \figref{fig:3d_shapes}.
\begin{figure}[ht]
    \centering
    \begin{subfigure}[t]{0.2\textwidth}
        \includegraphics[width=\textwidth]{Figures/datasets/box.png}
        \caption{Box}
    \end{subfigure}
    \begin{subfigure}[t]{0.2\textwidth}
        \includegraphics[width=\textwidth]{Figures/datasets/sphere.png}
        \caption{Sphere}
    \end{subfigure}
    \caption[3D shapes, generated by Ing. Lukáš Pospíšil, Ph.D. in POV-Ray]{3D shapes, generated by Ing. Lukáš Pospíšil, Ph.D. in POV-Ray \cite{povray}.}
    \label{fig:3d_shapes}
\end{figure}

\section{Cats and Dogs Dataset}
The Cats and Dogs dataset is created from the Oxford-IIIT Pet Dataset \cite{parkhi12a}. As the original dataset contains few damaged images, we use a fixed version of the dataset from \cite{ml4py_dataset}.


\chapter{Results}
In this chapter, we take a look at the different results of our classification.

\begin{table}
    \centering
    \input{tables/table.csv}
\end{table}


\chapter{Conclusion}
The goal of the thesis was to explore different feature extraction and classification techniques, which can be used for visual signals, i.e. images, classification. In the first part of the thesis, we introduced such techniques. In the next part of the thesis, we introduced three progressively complex datasets.

We attempted to classify these datasets, after reducing their dimension by applying different feature extraction techniques, using different classifiers. In the final part of the thesis, we show the results of this classification.

For the feature extraction, we make use of a global feature extractor, i.e. PCA, and three local feature extractors, namely SIFT, SURF, and ORB. The classification itself was carried out by the SVM model and the Bayesian model classifiers. We used a numerical solution of the Bayesian model applying the Spectral Projected Gradient method, and an analytical solution of the model exploiting the Jensen inequality.

We started by applying this classification pipeline to the simplest dataset, i.e. the dataset of images of 2D shapes. Afterward, we moved onto a more complex dataset of images of 3D shapes. The last dataset we attempt to classify is the most complex Cats and Dogs dataset.

We run our experiments on the Salomon supercomputer at IT4Innovations. The analytical solution of the Bayesian model classifier was the fastest to train while providing similar classification results to the SVM model in most cases. While feature extraction using the SURF extractor was always the fastest, the classification of such data led to worse results than the classification of the data extracted by SIFT or ORB. While the use of SIFT and ORB lead usually to similar classification results, ORB had the advantage of lower time complexity. In our experiments, the data extracted using PCA was not suitable for the classification.

The advantage of our approach to the classification problem, over the more popular neural networks, is our ability to understand and examine each step, and each parameter of each step. In the future, the hyperparameter optimization could be applied to the whole pipeline, from the extractor to the classifier parameters. It might also be beneficial to test different optimization algorithms in the SVM, allowing for faster convergence.


\printbibliography[heading=bibintoc]

\end{document}
