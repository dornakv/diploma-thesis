\chapter{Introduction}
Classification of visual data is an important task in the modern computer vision field. Nowadays, as the number of photographs taken every day increases, the importance of visual data classification is even more prominent. It can be used for data organization, automatic inspection in a manufacturing process, navigation, etc. \cite{wiki:Computer_vision}.

The popular approaches to visual data classification are based on the neural network \cite{da2017}. These complex models, which can consist of thousands of neurons, require a large amount of input data. Additionally, the meaning of each neuron and the influence on the performance of the overall model cannot be described precisely, therefore it is even impossible to interpret optimal parameters and a model as well.

In the bachelor thesis \cite{dornak2019}, we have shown that the classification of image data transformed into a feature space by the means of Scale Invariant Feature Transform (SIFT) feature transformation followed by the classification with the Support Vector Machine (SVM), is a reasonable option. In this thesis, we extend the idea to other feature transformation techniques and another conventional classification method.

For the feature transformation, we use and compare SIFT, Speed Up Robust Features (SURF), Oriented Fast and Rotated Brief (ORB), and Principal Component Analysis (PCA) extractors. These methods are described in details in Chapter \ref{sec:transformation}. For the construction of an optimal classification rule, we examine the commonly used SVM and the stochastic causality-based Bayesian model. See Chapter \ref{sec:classifiers} for details. We experiment with these approaches on three datasets of different complexity. The description of used benchmarks and our methodology is presented in Chapter \ref{sec:datasets}. The results of our comparison are discussed in Chapter \ref{sec:results}. Final Chapter \ref{sec:conclusion} concludes the thesis.
