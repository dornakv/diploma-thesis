\chapter{Introduction}
Classification of visual data is an important task in the modern computer vision field. Nowadays, as the number of photographs taken every day increases, the importance of visual data classification is even more prominent. It can be used for data organization, automatic inspection in a manufacturing process, navigation, etc. \cite{wiki:Computer_vision}.

The popular approach to visual data classification is the use of neural networks. However, we explore a different approach.

In \cite{dornak2019}, we have shown that the classification of image data, transformed into a feature space by the means of SIFT feature transformation, using the SVM, is a viable option. In this thesis, we expand the idea to other feature transformation techniques and another conventional classification method, the Bayesian Model. For the feature transformation, we use SIFT, SURF, ORB, and PCA extractors. We experiment with these approaches on three datasets of different complexity.


We describe the feature transformation techniques in chapter \ref{sec:transformation} and the classification methods in chapter \ref{sec:classifiers}. We introduce the three datasets of different complexity in chapter \ref{sec:datasets}. Finally, we present the results of this approach in chapter \ref{sec:results}.
